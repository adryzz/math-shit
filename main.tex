\documentclass{article}
\usepackage[fleqn]{mathtools}
\usepackage{amsfonts} 

\begin{document}
\title{Math Shit}
\boldmath

\section{Constants}

\begin{equation}
\lim_{x\to\pm\infty}(1+\frac{1}{x})^x=e\approx2.7182
\end{equation}

\section{Limits}

Definition of limit:
\begin{equation}
\lim_{x\to{x_0}}f(x) = l \textnormal{ if } \forall\varepsilon<0\exists I(x_0): |f(x) - l| < \varepsilon, \forall x \in I(x_0), x\neq x_0
\end{equation}

A vertical asymptote of a function happens in $x = c$ if:
\begin{equation}
\lim_{x\to{c^\pm}} f(x) = \pm\infty
\end{equation}

A horizontal asymptote of a function happens in $y = c$ if:
\begin{equation}
\lim_{x\to{\pm\infty}} f(x) = \pm c
\end{equation}

An oblique asymptote of a function happens if:
\begin{equation}
\lim_{x\to{\pm\infty}} \frac{f(x)}{x} = m
\end{equation}
with $m \neq 0$.
to find $q$: 
\begin{equation}
\lim_{x\to{\pm\infty}} [f(x) - mx] = q
\end{equation}

Addition of limits
\begin{equation}
\lim_{x\to{x_0}} [f(x) + g(x)] = \lim_{x\to{x_0}} f(x) + \lim_{x\to{x_0}} g(x)
\end{equation}

Product of limits
\begin{equation}
\lim_{x\to{x_0}} [f(x) \cdot g(x)] = \lim_{x\to{x_0}} f(x) \cdot \lim_{x\to{x_0}} g(x)
\end{equation}

Division of limits
\begin{equation}
\lim_{x\to{x_0}} [\frac{f(x)}{g(x)}] = \frac{\lim_{x\to{x_0}} f(x)}{\lim_{x\to{x_0}} g(x)}
\end{equation}

Power
\begin{equation}
\lim_{x\to{x_0}} f(x) = l \in \mathbb{R} \Rightarrow \lim_{x\to{x_0}} [f(x)]^n = l^n, \forall n\in \mathbb{N} - \{0\}
\end{equation}

Indeterminate forms:
\begin{equation}
+\infty - \infty, \infty \cdot 0, \frac{\infty}{\infty}, \frac{0}{0}, 1^{\infty}, 0^0, \infty^0
\end{equation}

Common limits:
\begin{equation}
\lim_{x\to0} \frac{\sin{x}}{x} = 1
\end{equation}
\begin{equation}
\lim_{x\to0} \frac{1-\cos{x}}{x} = 0
\end{equation}
\begin{equation}
\lim_{x\to0} \frac{1-\cos{x}}{x^2} = \frac{1}{2}
\end{equation}
\begin{equation}
\lim_{x\to\pm\infty}1+\frac{1}{x}^x=e
\end{equation}
\begin{equation}
\lim_{x\to0}\frac{\ln{(1+x)}}{x}=1
\end{equation}
\begin{equation}
\lim_{x\to0}\frac{e^x-1}{x}=1
\end{equation}
\begin{equation}
\lim_{x\to0}\frac{\log_a{(1+x)}}{x}=\log_a{e}
\end{equation}
\begin{equation}
\lim_{x\to0}\frac{a^x-1}{x}=\ln{a}
\end{equation}
\begin{equation}
\lim_{x\to0}\frac{(1+x)^k-1}{x}=k
\end{equation}

% Add discontinuity points, singularity points and how to check for shit

\section{Derivatives}

Definition of derivative:
\begin{equation}
D[f(x)] = f'(x) = \lim_{h\to0}(\frac{f(x+h)-f(x)}{h})
\end{equation}
(in $y=mx + q$, the derivative $f'(x)$ is the angular coefficient, or $m$)

Fundamental derivatives:
\begin{equation}
D [k] = 0
\end{equation}
\begin{equation}
D [ax] = a
\end{equation}
\begin{equation}
D [x^a] = ax^{a-1} \textnormal{ with } a \in \mathbb{R} \textnormal{ and } x > 0
\end{equation}
\begin{equation}
D [\sqrt{x}] = \frac{1}{2\sqrt{x}}
\end{equation}
\begin{equation}
D [\sin{x}] = \cos{x}
\end{equation}
\begin{equation}
D [\cos{x}] = -\sin{x}
\end{equation}
\begin{equation}
D [a^x] = a^x\cdot\ln{a}
\end{equation}
\begin{equation}
D [\log_a{x}] = \frac{1}{x}\cdot\log_a{e}
\end{equation}
When $a = e$:
\begin{equation}
D [e^x] = e^x
\end{equation}
\begin{equation}
D [\ln{x}] = \frac{1}{x}
\end{equation}

Operations with derivatives:
\begin{equation}
D [k\cdot f(x)] = k\cdot f'(x)
\end{equation}
\begin{equation}
D [f(x) + g(x)] = f'(x) + g'(x)
\end{equation}
\begin{equation}
D [f(x)\cdot g(x)] = f'(x) \cdot g(x) + f(x) \cdot g'(x)
\end{equation}
\begin{equation}
D [\frac{1}{f(x)}] = \frac{f'(x)}{f^2(x)}
\end{equation}
\begin{equation}
D [\frac{f(x)}{g(x)}] = \frac{f'(x) \cdot g(x) - f(x) \cdot g'(x)}{g^2(x)} \textnormal{ with } g(x) \neq 0
\end{equation}
\begin{equation}
D [\frac{f(x)}{g(x)}] = \frac{f'(x) \cdot g(x) - f(x) \cdot g'(x)}{g^2(x)} \textnormal{ with } g(x) \neq 0
\end{equation}
\begin{equation}
D [\tan{x}] = \frac{1}{\cos^2{x}} = 1 + \tan^2{x}
\end{equation}
\begin{equation}
D [\tan{x}] = \frac{1}{\cos^2{x}} = 1 + \tan^2{x}
\end{equation}
\begin{equation}
D [\cot{x}] = \frac{1}{\sin^2{x}} = -(1 + \cot^2{x})
\end{equation}
\begin{equation}
D [f(g(x))] = f'(g(x))\cdot g'(x)
\end{equation}
\begin{equation}
D [f(x)]^a = a[f(x)]^{a-1}\cdot f'(x)
\end{equation}
\begin{equation}
D [f^{-1}(y)] = \frac{1}{f'(x)} \textnormal{ with } x = f^{-1}(y)
\end{equation}
\begin{equation}
D [\arcsin{x}] = \frac{1}{\sqrt{\-x^2}}
\end{equation}
\begin{equation}
D [\arccos{x}] = -\frac{1}{\sqrt{\-x^2}}
\end{equation}
\begin{equation}
D [\arctan{x}] = \frac{1}{1+x^2}
\end{equation}
\begin{equation}
D [\arctan{x}] = -\frac{1}{1+x^2}
\end{equation}
Tangent line:
\begin{equation}
y - f(x_0) = f'(x0)\cdot (x - x_0)
\end{equation}
Given the function $y = f(x)$ and a point $x = c$, if $f'(c) = 0$, then $x = c$ is a stationary point, where the tangent is horizontal.

Differential:
\begin{equation}
\frac{\Delta y}{\Delta x} = \frac{f(x+h)-f(x)}{h}
\end{equation}

The differential of a function $f(x)$ relative to the point $x$ and to the $\Delta x$ increment, is the product of the derivative of the function, calculated in x, for the $\Delta x$ increment

\begin{equation}
\frac{\Delta y}{\Delta x} = \frac{f(x+h)-f(x)}{h}
\end{equation}

\begin{equation}
f'(x) = \frac{dy}{dx}
\end{equation}

Differentiability

% todo

Rolle's theorem

Given a function $f(x)$ continuous in an interval $[a;b]$, differentiable in $(a; b)$, where $f(a) = f(b)$, then there exists at least a point $f(c) \in [a; b]$ where $f'(c) = 0$

Lagrange's theorem

Given a function $f(x)$ continuous and differentiable in an interval $[a;b]$, then there exists a point where the following relationship is true:
\begin{equation}
\frac{f(b)-f(a)}{b-a} = f'(c)
\end{equation}

If a function $f(x)$ continuous in an interval $[a;b]$, differentiable in $(a; b)$, where $f'(x) = 0$ in every point within the interval, then $f(x)$ has the same value across the whole interval.

Cauchy's theorem

If two functions $f(x)$ and $g(x)$ are continuous and differentiable in an interval $[a;b]$, and $g'(x)\neq 0 \forall x \in [a;b]$, then there exists at least a point where the following is true:
\begin{equation}
\frac{f(b)-f(a)}{g(b)-g(a)} = \frac{f'(c)}{g'(c)}
\end{equation}

De L' Hospital's theorem

Given two functions $f(x)$ and $g(x)$ continuous and differentiable in an open interval $I$ except possibly at a point c contained in $I$, if $lim_{x\to c} f(x) = \lim_{x\to c} g(x) = 0$ or $\pm\infty$ and $g'(x)\neq 0 \forall x \in I$ with $x\neq c$ and $\lim_{x\to c} \frac{f'(x)}{g'(x)}$ exists then
\begin{equation}
lim_{x\to c}\frac{f(x)}{g(x)} = lim_{x\to c}\frac{f'(x)}{g'(x)}
\end{equation}

\section{Integrals}

\begin{equation}
\int_{a}^{b} f(x) dx = [F(x)]_{a}^{b} = F(b) - F(a)
\end{equation}

\section{Trigonometry}

The main law of trigonometry:
\begin{equation}
\sin^2{\alpha} + \cos^2{\alpha} = 1
\end{equation}

Other functions:
\begin{equation}
\tan{\alpha} = \frac{\sin{\alpha}}{\cos{\alpha}}
\end{equation}
\begin{equation}
\cot{\alpha} = \frac{\cos{\alpha}}{\sin{\alpha}}
\end{equation}
\begin{equation}
\csc{\alpha} = \frac{1}{\sin{\alpha}}
\end{equation}
\begin{equation}
\sec{\alpha} = \frac{1}{\cos{\alpha}}
\end{equation}

\end{document}