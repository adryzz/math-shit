\documentclass{article}
\usepackage[fleqn]{mathtools}
\begin{document}
\title{Math Shit}
\boldmath

\section{Constants}

\begin{equation}
\lim_{x\to\infty}(1+\frac{1}{x})^x=e\approx2.7182
\end{equation}

\section{Limits}

Definition of limit:
\begin{equation}
\lim_{x\to{x_0}}f(x) = l if \forall\varepsilon<0\exists I(x_0): |f(x) - l| < \varepsilon, \forall x \in I(x_0), x\neq x_0
\end{equation}

A vertical asymptote of a function happens in $x = c$ if:
\begin{equation}
\lim_{x\to{c^\pm}} f(x) = \pm\infty
\end{equation}

A horizontal asymptote of a function happens in $y = c$ if:
\begin{equation}
\lim_{x\to{\pm\infty}} f(x) = \pm c
\end{equation}

An oblique asymptote of a function happens if:
\begin{equation}
\lim_{x\to{\pm\infty}} \frac{f(x)}{x} = m
\end{equation}
with $m \neq 0$.

\section{Derivatives}

Definition of derivative:
\begin{equation}
D[f(x)] = f'(x) = \lim_{h\to0}(\frac{f(x)-f(x+h)}{h})
\end{equation}

\section{Integrals}

\begin{equation}
\int_{a}^{b} f(x) dx = [F(x)]_{a}^{b} = F(b) - F(a)
\end{equation}

\end{document}